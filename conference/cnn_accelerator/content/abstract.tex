\begin{abstract}
Convolutional neural networks (CNNs) have become ubiquitous in the field of image processing, computer vision, and artificial intelligence (AI). Given the high computational demands of CNNs, dedicated hardware accelerators have been implemented to improve compute efficiency in FPGAs and ASICs. However, most commercial general-purpose deep learning processing units (DPUs) struggle with support for low-power, resource-limited embedded devices.
In this publication, we present a dedicated hardware accelerator for TensorFlow (TF) Lite on embedded FPGA emulating an Edge TPU coprocessor to delegate Conv2D and DepthwiseConv2D tensor operations. The hardware design is implemented with high-level synthesis (HLS). This accelerator incorporates the support for TF Lite quantization for fixed-point and floating-point. The proposed optimization decomposes floating-point calculation for the dot-product. This approach accelerates computation, reduces energy consumption and resource utilization. To demonstrate the potential of the proposed accelerator, we address a design exploration with custom-built CNNs covering fixed-point quantization, floating-point single precision, half-precision, brain floating-point, TensorFloat, and custom reduced formats for approximate processing, including logarithmic computation. A single accelerator running at 150 MHz on a Xilinx Zynq-7020 achieves 45X runtime acceleration on Conv2D tensor operation compared with ARM Cortex-A9 at 666MHz, and 5X compared with the equivalent standard floating-point implementation in HLS with Xilinx LogiCORE IP. This accelerator yields a peak performance of 1.6 TFLOPS/watt and 152 MFLOP/s. The entire hardware design and the implemented TF Lite software extensions are available as an open-source project.
\end{abstract}

\begin{IEEEkeywords}
Artificial intelligence, convolutional neural networks, depthwise separable convolution, hardware accelerator, TensorFlow Lite, embedded systems, FPGA, custom floating-point, logarithmic computation, approximate computing
\end{IEEEkeywords}
