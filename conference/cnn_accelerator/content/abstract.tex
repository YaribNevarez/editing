\begin{abstract}
Convolutional neural networks (CNNs) have become ubiquitous in the field of image processing, computer vision, and artificial intelligence (AI). Given the high computational demands of CNNs, dedicated hardware accelerators have been implemented to improve compute performance in FPGAs and ASICs. However, most commercial general-purpose deep learning processing units (DPUs) struggle with support for low-power, resource-limited embedded devices.
In this paper, we present a tensor processor (TP) as a dedicated hardware accelerator for TensorFlow (TF) Lite on embedded FPGA. We accelerate Conv2D and DepthwiseConv2D tensor operations with fixed-point and floating-point. The proposed compute optimization performs vector dot-product with hybrid custom floating-point and logarithmic approximation. This approach accelerates computation, reduces energy consumption and resource utilization. To demonstrate the potential of the proposed architecture, we address a design exploration with four compute engines: (1) fixed-point, (2) Xilinx floating-point LogiCORE IP, (3) hybrid custom floating-point approximation, and (4) hybrid logarithmic approximation.
The hardware design is implemented with high-level synthesis (HLS). A single TP running at 150 MHz on a Xilinx Zynq-7020 achieves 45X runtime acceleration and 951X power reduction on Conv2D tensor operation compared with ARM Cortex-A9 at 666MHz, and 4.5X compared with the equivalent implementation with floating-point LogiCORE IP. The entire hardware design and the implemented TF Lite software extensions are available as an open-source project.
\end{abstract}

\begin{IEEEkeywords}
Artificial intelligence, convolutional neural networks, depthwise separable convolution, hardware accelerator, TensorFlow Lite, embedded systems, FPGA, custom floating-point, logarithmic computation, approximate computing
\end{IEEEkeywords}
