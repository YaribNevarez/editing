\section{Experimental results}
\label{sec:experimental_results}
The proposed hardware/software framework is demonstrated on a Xilinx Zynq-7020 SoC (Zybo-Z7 development board). On the PL, we implement the proposed hardware architecture with a clock frequency at $150 MHz$. On the PS, we execute TF Lite Micro (bare-metal) on the ARM Cortex-A9 at $666MHz$ with NEON floating-point unit (FPU)\cite{xilinx2015zynq}.

To evaluate the performance, we build models $A$ and $B$ in TensorFlow, see \Fig{fig:models}. To evaluate \emph{DConv} tensor operation, model $B$ incorporates depthwise separable convolution operations (a depthwise convolution followed by
a pointwise convolution).

$A$ and $B$ are evaluated with the following hardware implementations:
(1) fixed-point, (2) floating-point LogiCORE, (3) hybrid custom floating-point approximation, and (4) hybrid logarithmic approximation.


\begin{figure}[t!]
	\centering
	\includegraphics[width=0.5\textwidth]{../figures/models.pdf}
	\caption{CNN-based models for case study.}
	\label{fig:models}
\end{figure}

\subsection{Hardware implementations}
\begin{enumerate}
\item{Fixed-point}: To evaluate the compute performance on fixed-point, we convert $A$ and $B$ to TF Lite models with 8-bit fixed-point quantization. The compute performance is presented in \tab{tab:performance_fixed_point}. A runtime execution of $A$ is illustrated in \fig{fig:sched_model_a_fixed}. This implementation achieves a peak acceleration of $45.23\times$ in model $A$ at the tensor operation \emph{(4A) Conv}, see \tab{tab:performance_fixed_point}.

\item{Floating-point LogiCORE}: To evaluate the compute performance on floating-point models, we convert $A$ and $B$ to TF Lite without quantization. The compute performance is presented in \tab{tab:performace_float_logicore}.
This implementation achieves a peak acceleration of $9.77\times$ in model $A$ at the tensor operation \emph{(4A) Conv}.

\item{Hybrid custom floating-point approximation}: This implementation presents a peak acceleration of $44.87\times$ in model $A$ at the tensor operation \emph{(4A) Conv}. See \tab{tab:performace_float_hybrid}. This implementation achieves a $4.59\times$ acceleration over the LogiCORE floating-point implementation. The runtime execution of model $B$ with \emph{DConv} tensor operations is illustrated in \fig{fig:sched_model_b_float}.

\item{Hybrid logarithmic approximation}: This implementation is presented for comparison in \fig{fig:sched_model_a_float}, which shows the runtime executions of model $A$ with the proposed floating-point solutions including hybrid logarithmic approximation.
\end{enumerate}

\begin{table}[!htp]\centering
	\caption{Compute performance with fixed-point on model $A$ and $B$.}\label{tab:performance_fixed_point}
	\scriptsize
\begin{tabular}{lrrrrrrr}\toprule
	\multicolumn{2}{c}{\textbf{Tensor operation}} &\textbf{CPU} &\multicolumn{3}{c}{\textbf{TP (fixed-point)}} &\multirow{2}{*}{\textbf{Accel.}} \\\cmidrule{1-6}
	\textbf{Operation} &\textbf{MOP} &\textbf{t (ms)} &\textbf{t (ms)} &\textbf{MOP/s} &\textbf{GOP/W} & \\\midrule
	\multicolumn{2}{c}{\textbf{Model $A$}} & & & & & \\
	(1A) Conv &1.769 &700.22 &55.19 &32.06 &0.23 &\textbf{12.69} \\
	(2A) Conv &37.748 &12,666.91 &297.08 &127.06 &0.93 &\textbf{42.64} \\
	(3A) Conv &18.874 &6,081.01 &142.99 &131.99 &0.97 &\textbf{42.53} \\
	(4A) Conv &18.874 &5,543.77 &122.58 &153.97 &1.13 &\textbf{45.23} & \\\midrule
	\multicolumn{2}{c}{\textbf{Model $B$}} & & & & & \\
	(1B) DConv &0.027 &13.43 &0.63 &43.74 &0.25 &\textbf{21.25} \\
	(2B) Conv &0.196 &129.95 &11.57 &16.98 &0.12 &\textbf{11.23} \\
	(3B) DConv &0.147 &69.18 &3.33 &44.26 &0.25 &\textbf{20.77} \\
	(4B) Conv &1.048 &378.78 &9.96 &105.25 &0.77 &\textbf{38.02} \\
	(5B) Conv &2.359 &694.60 &16.46 &143.22 &1.05 &\textbf{42.20} \\
	\bottomrule
\end{tabular}
\end{table}

\begin{figure}[t!]
	\centering
	\includegraphics[width=0.5\textwidth]{../figures/sched_A_fixed_point.pdf}
	\caption{Compute performance with fixed-point on model $A$.}
	\label{fig:sched_model_a_fixed}
\end{figure}


\begin{table}[!htp]\centering
	\caption{Compute performance with floating-point LogiCORE on models $A$ and $B$.}\label{tab:performace_float_logicore}
	\scriptsize
	\begin{tabular}{lrrrrrrr}\toprule
		\multicolumn{2}{c}{\textbf{Tensor operation}} &\textbf{CPU} &\multicolumn{3}{c}{\textbf{TP (floating-point LogiCORE)}} &\multirow{2}{*}{\textbf{Accel.}} \\\cmidrule{1-6}
		\textbf{Operation} &\textbf{MOP} &\textbf{t (ms)} &\textbf{t (ms)} &\textbf{MOP/s} &\textbf{GOP/W} & \\\midrule
		\multicolumn{2}{c}{\textbf{Model $A$}} & & & & & \\
		(1A) Conv &1.769 &670.95 &120.07 &14.73 &0.21 &\textbf{5.59} \\
		(2A) Conv &37.748 &12,722.13 &1,328.08 &28.42 &0.40 &\textbf{9.58} \\
		(3A) Conv &18.874 &6,094.85 &636.53 &29.65 &0.42 &\textbf{9.58} \\
		(4A) Conv &18.874 &5,564.79 &569.30 &33.15 &0.47 &\textbf{9.77} & \\\midrule
		\multicolumn{2}{c}{\textbf{Model $B$}} & & & & & \\
		(1B) DConv &0.027 &11.51 &1.557 &17.75 &0.23 &\textbf{7.39} \\
		(2B) Conv &0.196 &94.82 &20.487 &9.59 &0.13 &\textbf{4.62} \\
		(3B) DConv &0.147 &58.84 &8.355 &17.64 &0.23 &\textbf{7.04} \\
		(4B) Conv &1.048 &368.66 &40.271 &26.03 &0.37 &\textbf{9.15} \\
		(5B) Conv &2.359 &697.08 &72.981 &32.32 &0.46 &\textbf{9.55} \\
		\bottomrule
	\end{tabular}
\end{table}

\begin{table}[!htp]\centering
	\caption{Compute performance with hybrid custom floating-point approximation on models $A$ and $B$.}\label{tab:performace_float_hybrid}
	\scriptsize
\begin{tabular}{lrrrrrrr}\toprule
	\multicolumn{2}{c}{\textbf{Tensor operation}} &\textbf{CPU} &\multicolumn{3}{c}{\textbf{TP (H. custom floating-point)}} &\multirow{2}{*}{\textbf{Accel.}} \\\cmidrule{1-6}
	\textbf{Operation} &\textbf{MOP} &\textbf{t (ms)} &\textbf{t (ms)} &\textbf{MOP/s} &\textbf{GOP/W} & \\\midrule
	\multicolumn{2}{c}{\textbf{Model $A$}} & & & & & \\
	(1A) Conv &1.769 &670.95 &68.50 &25.83 &0.39 &\textbf{9.8} \\
	(2A) Conv &37.748 &12,722.13 &307.83 &122.63 &1.85 &\textbf{41.33} \\
	(3A) Conv &18.874 &6,094.85 &147.97 &127.55 &1.93 &\textbf{41.19} \\
	(4A) Conv &18.874 &5,564.79 &124.03 &152.17 &2.30 &\textbf{44.87} & \\\midrule
	\multicolumn{2}{c}{\textbf{Model $B$}} & & & & & \\
	(1B) DConv &0.027 &11.51 &1.41 &19.63 &0.27 &\textbf{8.17} \\
	(2B) Conv &0.196 &94.82 &20.34 &9.43 &0.14 &\textbf{4.66} \\
	(3B) DConv &0.147 &58.84 &6.58 &22.41 &0.31 &\textbf{8.94} \\
	(4B) Conv &1.048 &368.66 &12.75 &82.23 &1.24 &\textbf{28.91} \\
	(5B) Conv &2.359 &697.08 &17.14 &137.68 &2.08 &\textbf{40.68} \\
	\bottomrule
\end{tabular}
\end{table}


\begin{figure*}[t!]
	\centering
	\includegraphics[width=\textwidth]{../figures/sched_A_float_all.pdf}
	\caption{Compute performance with the proposed floating-point solutions on model $A$.}
	\label{fig:sched_model_a_float}
\end{figure*}


\begin{figure}[t!]
	\centering
	\includegraphics[width=0.5\textwidth]{../figures/sched_B_float.pdf}
	\caption{Compute performance on model $B$ (floating-point).}
	\label{fig:sched_model_b_float}
\end{figure}

\subsection{Classification accuracy}

We evaluate the classification accuracy of the CNN models under the effects of custom floating and logarithmic quantization. \tab{tab:formats} presents the list of custom formats proposed for evaluation. In this case, the \emph{filter} and \emph{bias} tensors are quantized from base floating-point representation (IEEE 754) into custom reduced formats with bit-truncation and -rounding methods. For this evaluation, we train $A$ an $B$ for image classification with CIFAR-10 dataset. We deploy the models with a baseline accuracy of 76.6\% for $A$, and 68.8\% for $B$. See \fig{fig:accuracy}.

\begin{table}[!htp]\centering
	\caption{Implemented floating-point formats for accuracy evaluation.}\label{tab:formats}
	\scriptsize
	\begin{tabular}{lrrrrr}\toprule
		\multicolumn{5}{c}{\textbf{Floating-point formats}} \\\cmidrule{1-5}
		\textbf{Name} &\textbf{Size (bits)} &\textbf{Sign} &\textbf{Exponent} &\textbf{Mantissa} \\\midrule
		Logarithmic &6 &1 &5 &0 \\
		S1-E5-M1 &7 &1 &5 &1 \\
		S1-E5-M2 &8 &1 &5 &2 \\
		S1-E5-M3 &9 &1 &5 &3 \\
		S1-E5-M4 &10 &1 &5 &4 \\
		Float16 &16 &1 &5 &10 \\
		BFloat16 &16 &1 &8 &7 \\
		Tensor Float &19 &1 &8 &10 \\
		Float32 &32 &1 &8 &23 \\
		\bottomrule
	\end{tabular}
\end{table}

\begin{figure}[t!]
	\centering
	\includegraphics[width=0.5\textwidth]{../figures/all_models_accuracy.pdf}
	\caption{Accuracy performance using hybrid custom floating-point approximation with various formats. Samples: CIFAR-10 test dataset ($10,000$ images).}
	\label{fig:accuracy}
\end{figure}

\subsection{Resource utilization and power dissipation}
The resource utilization and power dissipation of the TP is listed in \tab{tab:resource}. The power dissipation of the Zynq device is presented in \fig{fig:power}. 

\begin{table}[!htp]\centering
	\caption{Resource utilization and power dissipation of the proposed TP engines.}\label{tab:resource}
	\scriptsize
\begin{tabular}{lrrrrrr}\toprule
	\textbf{} &\multicolumn{4}{c}{\textbf{Post-implementation resource utilization}} &\multirow{2}{*}{\textbf{Power (W)}} \\\cmidrule{2-5}
	\textbf{TP engine} &\textbf{LUT} &\textbf{FF} &\textbf{DSP} &\textbf{BRAM 18K} & \\\midrule
	\multicolumn{6}{l}{\textbf{1) Fixed-point}} \\
	Conv &5,677 &4,238 &78 &70 &0.136 \\
	DConv &7,232 &5,565 &106 &70 &0.171 \\
	Conv + DConv &12,684 &8,015 &160 &70 &0.248 & \\\midrule
	\multicolumn{6}{l}{\textbf{2) Floating-point LogiCore}} \\
	Conv &4,670 &3,909 &59 &266 &0.070 \\
	DConv &6,263 &5,264 &82 &266 &0.075 \\
	Conv + DConv &10,871 &7,726 &123 &266 &0.119 \\\midrule
	\multicolumn{6}{l}{\textbf{3 ) Hybrid custom floating-point approximation}} \\
	Conv &6,787 &4,349 &56 &74 &0.066 \\
	DConv &8,209 &5,592 &79 &74 &0.072 \\
	Conv + DConv &14,590 &8,494 &117 &74 &0.108 & \\\midrule
	\multicolumn{6}{l}{\textbf{4) Hybrid logarithmic approximation}} \\
	Conv &6,662 &4,242 &54 &58 &0.060 \\
	DConv &8,110 &5,380 &77 &58 &0.066 \\
	Conv + DConv &14,370 &8,175 &113 &58 &0.105 \\
	\bottomrule
\end{tabular}
\end{table}

\begin{figure}[t!]
	\centering
	\includegraphics[width=0.5\textwidth]{../figures/power_breackdown.pdf}
	\caption{Estimated power dissipation of the Zynq-7020 SoC with different TP engines.}
	\label{fig:power}
\end{figure}


\subsection{Discussion}
\begin{enumerate}
	\item{Acceleration}: The dot-product with hybrid custom floating-point approximation achieves better performance with larger vectors. Since \emph{DConv} performs a channel-wise spatial dot-product, this implementation presents very limited acceleration compared with the LogiCORE-based implementation.
	
	\item{Energy}: The implementations with hybrid custom floating-point and logarithmic approximation are the most efficient with energy reduction of $954\times$ and $1,055\times$, respectively. \tab{tab:edp} presents the energy-delay product (EDP) and energy reduction in \emph{(4A) Conv} operator.
	
\begin{table}[!htp]\centering
	\caption{Energy consumption in tensor operation \emph{(4A) Conv}.}\label{tab:edp}
	\scriptsize
	\begin{tabular}{lrrrrr}\toprule
		\textbf{Engine} & \textbf{t (ms)} &\textbf{Power (W)} &\textbf{EDP (J)} &\textbf{Reduction} \\\midrule
		CPU &5,564.79 &1.404 &7,812.97 &1.00 \\
		Fixed-point &122.58 &0.136 &16.67 &468.66 \\
		Floating-point LogiCORE &569.30 &0.070 &39.85 &196.05 \\
		Hybrid custom floating-point &124.03 &0.066 &8.19 &\textbf{954.43} \\
		Hybrid logarithmic &123.32 &0.060 &7.40 &\textbf{1,055.92 }\\
		\bottomrule
	\end{tabular}
\end{table}
	
	\item{Resource utilization}: In the case of fixed-point implementation, the TP includes additional logic to support the quantized multiplier and shifter channel-wise corrections implemented for TF Lite 8-bit quantization. This TP presents the highest power dissipation.
	
	\item{Accuracy}: Based on the presented classification accuracy, the hybrid custom floating-point approximation presents the best trade off between QoR and energy-efficiency.
	
	Theres is no accuracy degradation when using fixed-point and floating-point LogiCORE IP.
\end{enumerate}