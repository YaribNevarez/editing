\documentclass[conference]{IEEEtran}
\IEEEoverridecommandlockouts
% The preceding line is only needed to identify funding in the first footnote. If that is unneeded, please comment it out.
\usepackage{cite}
\usepackage{amsmath,amssymb,amsfonts}
\usepackage{algorithmic}
\usepackage{graphicx}
\usepackage{textcomp}
\usepackage{xcolor}
\usepackage[a4paper, total={184mm,239mm}]{geometry}
\def\BibTeX{{\rm B\kern-.05em{\sc i\kern-.025em b}\kern-.08em
    T\kern-.1667em\lower.7ex\hbox{E}\kern-.125emX}}

%\usepackage{algorithm}
\usepackage{algorithmic}
\usepackage[ruled, vlined, linesnumbered]{algorithm2e}
\usepackage{graphicx}
\usepackage{textcomp}

\usepackage{enumitem}

%Please add the following packages if necessary:
\usepackage{booktabs, multirow} % for borders and merged ranges
\usepackage{soul}% for underlines
%\usepackage[table]{xcolor} % for cell colors
\usepackage{changepage, threeparttable} % for wide tables


\usepackage[commentmarkup=footnote,final]{changes} % Add changes for correction in text. USE THIS FOR FINAL VERSION
%\usepackage[commentmarkup=footnote]{changes} % Add changes for correction in text. USE THIS FOR REVIEW

\newcommand\Fig[1]{\textbf{Fig.}~\ref{#1}}
\newcommand\fig[1]{\textbf{Fig.}~\ref{#1}}
\newcommand\Tab[1]{\textbf{Tab.}~\ref{#1}}
\newcommand\tab[1]{\textbf{Tab.}~\ref{#1}}
\newcommand\Equ[1]{\textbf{Eq.}~(\ref{#1})}
\newcommand\equ[1]{\textbf{Eq.}~(\ref{#1})}
\newcommand\Sect[1]{\textbf{Sec.}~\ref{#1}}
\newcommand\sect[1]{\textbf{Sec.}~\ref{#1}}
\newcommand\Refs[1]{\textbf{Ref.}~\cite{#1}}
\newcommand\refs[1]{\textbf{Ref.}~\cite{#1}}
\newcommand\Algo[1]{\textbf{Algorithm}~\ref{#1}}
\newcommand\algo[1]{\textbf{Algorithm}~\ref{#1}}


\newcommand\REVIEW[1]{{#1}}

\begin{document}

%\title{Accelerating Convolutional Neural Networks for TensorFlow Lite on Embedded FPGA %with Custom Floating-Point Approximation}

\title{Accelerator Framework for Mapping Floating-Point CNN on Low-Power Resource-Limited FPGAs}

\author{
\IEEEauthorblockN{1\textsuperscript{st} }
\IEEEauthorblockA{\textit{dept. name of organization (of Aff.)} \\
\textit{name of organization (of Aff.)}\\
City, Country \\
email address or ORCID}
\and
\IEEEauthorblockN{2\textsuperscript{nd} }
\IEEEauthorblockA{\textit{dept. name of organization (of Aff.)} \\
\textit{name of organization (of Aff.)}\\
City, Country \\
email address or ORCID}
\and
\IEEEauthorblockN{3\textsuperscript{rd} }
\IEEEauthorblockA{\textit{dept. name of organization (of Aff.)} \\
\textit{name of organization (of Aff.)}\\
City, Country \\
email address or ORCID}
}
\maketitle
\begin{abstract}
In this article, we present a design exploration framework for floating-point convolutional neural networks (CNNs) acceleration on low-power, resource-limited embedded FPGAs targeting IoT sensor data analytic applications. We propose a scalable hardware architecture with customizable tensor processors (TPs) integrated with TensorFlow Lite. The implemented hardware optimization realizes hybrid custom floating-point and logarithmic dot-product approximation. This approach accelerates computation, reduces energy consumption and resource utilization while maintaining inference accuracy. Experimental results on MiniZed (XC7Z007S) and Zybo (XC7Z010) demonstrate peak acceleration and power efficiency of 105X and 5.5 GFLOP/s/W, respectively.
\end{abstract}

\begin{IEEEkeywords}
Artificial intelligence, convolutional neural networks, depthwise separable convolution, hardware accelerator, TensorFlow Lite, embedded systems, FPGA, custom floating-point, logarithmic computation, approximate computing
\end{IEEEkeywords}


\section{Introduction}
\label{sec:introduction}
%%% General intro
\PARstart{A}{rtificial} Intelligence (AI) is increasingly attracting the interest of industry and academia; in particular,  Artificial Neural Networks (ANNs). Historically, ANNs can be classified into three different generations \cite{Design_Exploration_SbS_Trans20}: the first one is represented by the classical McCulloch and Pitts neuron model using discrete binary values as outputs; the second one is represented by more complex architectures as Multi-Layer Perceptrons and Convolutional Neural Networks (CNN) using continuous activation functions; while the third generation is represented by Spiking Neural Networks (SNNs) using spikes as means for information exchange between groups of neurons. Although the AI field is currently dominated by Deep Neural Networks (DNN) from the second generation, nowadays the SNNs belonging to the third generation are receiving considerable attention \cite{Spinnaker_Trans13,ernst2007efficient,Design_Exploration_SbS_Trans20, SNN_Survey_Trans19} due to their advantages in terms of robustness and the
potential to achieve a power efficiency close to that of the human
brain (see section~\ref{sec:sbs} for more details).

%%% SbS intro
Among the family of SNNs, the SbS neural network \cite{ernst2007efficient} is inspired by the natural computing of the mammalian brain, being a biologically plausible approach although with less complexity than other SNNs. The SbS model differs fundamentally from conventional ANNs since (a) the building block of the network are inference populations (IP) which are an optimized generative representation with non-negative values, (b) time progresses from one spike to the next, preserving the property of stochastically firing neurons, and (c) a network has only a small number of parameters, which is an advantageous stochastic version of Non-Negative Matrix Factorization (NNMF), which is noise-robust and easy to handle. In regard to biological realism and computational effort to simulate neural networks, these properties place the SbS network in between non-spiking NN and stochastically spiking NN \cite{rotermund2019Backpropagation}.

%%%%%%% Problem statement
Although SbS networks provide numerous advantages over traditional ANNs and CNNs, their computational cost and memory footprint are the fundamental limitations for efficient deployment on resource-limited devices. As a newly emerging SNN algorithm, most SbS models use floating-point computation, which imposes a high cost on processing and memory footprint. Model quantization has the potential to improve computational performance on resource-limited devices; however, this solution is often accompanied by quantization-aware training methods that, in some cases, are problematic or even inaccessible, particularly in SNN algorithms\cite{zhang2018survey}.

%%%%%%% Contributions
In this paper, we present a hardware architecture for SbS network based on approximate dot-product computation using hybrid custom floating-point and logarithmic number representation. We use approximate computing as a design paradigm to leverage the intrinsic resilience of SbS networks to execute computations approximately, leading to higher efficiency and performance enhancement. This approach represents an alternative for efficient deployment of non-quantized SbS models on resource-limited devices. Our main contributions are as follows:

\begin{itemize}
	\item We develop a hardware module for approximate dot-product computation. The approximate dot-product consist of (1) the element-wise multiplication is done by adding integer exponents as well as accumulation is done by adding denormalized integer products, which increases computational throughput, (2) the synaptic weight vector uses either reduced custom floating-point or logarithmic representation, which reduces memory footprint, and (3) the neuron vector uses either standard or custom floating-point representation, which preserves inference accuracy.
	\item We address a design exploration evaluating computational latency, accuracy degradation, noise robustness, resource utilization and power dissipation. Experimental results show that the proposed architecture achieve 20.5x latency enhancement with less than 0.5\% of accuracy degradation.
	\item We present a noise robustness study using positive additive uniformly distributed noise on the input images. Experimental results show that the SbS network simulation presents an accuracy degradation of less than 1\% using custom floating-point, and less than 5\% using logarithmic computation.
	\item Our proposed hardware module for approximate dot-product computation is adaptable for other neural networks. This represents an alternative for efficient deployment of non-quantized floating-point neural network models on resource-limited devices.
\end{itemize}


The rest of the paper is organized as follows. Section~\ref{sec:related_work} covers the related work; Section~\ref{sec:background} introduces the background; Section~\ref{sec:system_design} describes the system design; Section~\ref{sec:experimental_results} presents the experimental results; Section~\ref{sec:conclusions} concludes the paper.


To promote the research on SbS, the entire framework is made available to the public as an open-source project at http://www.ids.uni-bremen.de/sbs-framework.html


\section{Related work}
\label{sec:related_work}
%%%%%%%%%%%%%%%%%%%%%%%%%%%%%%%%%%%%%%%%%%%%%%%%%%%%%%
Recently, some state of the art survey on hardware architectures for SNN have been reported \cite{Design_Exploration_SbS_Trans20, SNN_Survey_Trans19}. In particular, Nassim Abderrahmane et al. briefly describe and compare some recent implementations of ASIC and FPGA where only two are suitable for embedded systems. As a typical example of the current state of the art, Furber et al., presents SpiNNaker \cite{Spinnaker_Trans13}, aiming to simulate very large SNNs in real-time. It is composed of 48 chips containing a shared memory and 18 ARM cores with small local memory each processor. The main feature of SpiNNaker are the support for several neuron models, synaptic plasticity rules, incremental learning capabilities and efficient communication system. This architecture is suitable for neuroscience research but not for embedded applications. Further on, in a previous research Rotermund et al., demonstrated the feasibility of a neuromorphic SbS IP in a Xilinx Virtex 6 FPGA \cite{rotermund2018massively}. It provides a massively parallel architecture, optimized for memory access and suitable for ASIC implementations. However, this design is considerably resource-demanding to be deployed as a full and functional SbS network in the current embedded technology.

Beside the actual architectures, researches have also identified design methodologies as a critical problem for the efficient development of SNN \cite{Design_Exploration_SbS_Trans20}. For example, Nassim Abderrahmane et al., develop a behavioral level simulator for neuromorphic hardware architectural exploration named NAXT, capable to reduce the number of spikes while keeping the neuron's model resulting in lower power consumption. This work provides a great exploration of SNN for different network topologies and computation approaches on NAXT. However, this simulator presents 62\% of accuracy on MNIST classification task.
%%%%%%%%%%%%%%%%%%%%%%%%%%%%%%%%%%%%%%%%%%%%%%%%%%%%%%
\section{Background}
\label{sec:background}
\subsection{Conv2D tensor operation}

 	\begin{eqnarray} \label{eq:Conv2D}
 	Conv2D\left(W,y\right)=\sum_{k,l,m}^{K,L,M}W_{(k,l,m)} \cdot y_{(i+k,j+l,m)}
 	\end{eqnarray} 	
 	
\subsection{DepthwiseConv2D tensor operation}

 	\begin{eqnarray} \label{eq:DepthwiseConv2D}
 	DepthwiseConv2D\left(W,y\right)=\sum_{k,l}^{K,L}W_{(k,l)} \odot y_{(i+k,j+l)}
 	\end{eqnarray}
\section{System Design}
\label{sec:system_design}

\REVIEW{
	In this section, we revise the system design of \mbox{\cite{nevarez2020accelerator}}. In Ref. \mbox{\cite{nevarez2020accelerator}}, we presented a scalable hardware architecture composed of generic homogeneous accelerator units (AUs). This design works entirely with standard floating-point arithmetic (IEEE 754), which represents an unnecessary overhead for error-resilient applications. Furthermore, this architecture does not implement stationary synaptic weight matrix in the hardware AUs, resulting in heavy data movement and longer computational latency.
	
	In this publication, we present an enhanced hardware architecture composed of specialized heterogeneous processing units (PUs) with hybrid custom floating-point and logarithmic dot-product approximation. This approach represents an advantageous design for error-resilient applications in resource-constrained devices due to the reduced computational costs and memory footprint. Furthermore, the proposed approach allows the implementation of stationary synaptic weight matrices. These novelties result in an improved overall system design.
	}

Regarding the software architecture, this is structured as a
layered object-oriented application framework written in the C programming language. This offers a comprehensive high level embedded software application programming interface (API) that allows the construction of scalable sequential SbS networks with configurable hardware acceleration. Conceptually this design is modular, reusable, and extensible. The overall structure is depicted in \fig{fig:sw_stack}.

\begin{figure}[t!]
	\centering
	\includegraphics[width=0.5\textwidth]{../figures/sbs_software_component.pdf}
	\caption{System-level overview of the embedded software architecture.}
	\label{fig:sw_stack}
\end{figure}

\subsection{Hardware architecture} \label{Hardware_architecture}
As a hardware/software co-design, the system architecture is an embedded CPU+FPGA-based platform, where the acceleration of SbS network computation is based on asynchronous\footnote{The system is synchronous at the circuit level, but the execution is asynchronous in terms of jobs.} execution \replaced{in}{of} parallel heterogeneous processing units: \emph{Spike} (input layer), \emph{Conv} (convolution), \emph{Pool} (pooling), and \emph{FC} (fully connected). \fig{fig:hw_sbs} illustrates the system \replaced{hardware architecture}{overview} as a scalable structure. For \REVIEW{hyperparameter} configuration, each PU uses AXI-Lite interface. For data transfer, each PU uses AXI-Stream interfaces via Direct Memory Access (DMA) allowing data movement with high transfer rate. Each PU asserts an interrupt flag once the job or transaction is complete. This interrupt event is handled by the embedded CPU to collect results and start a new transaction.

The hardware architecture can resize its resource utilization by changing the number of PUs instances \REVIEW{prior to the hardware synthesis}, this provides scalability with a good trade-off between area and throughput. The dedicated PUs for \emph{Conv} and \emph{FC} implement the proposed dot-product approximation as a system component. The PUs are written in C using Vivado HLS (High-Level Synthesis) tool. In this publication, we illustrate the integration of the approximate dot-product component on the \emph{Conv} processing unit.

\begin{figure}[t!]
	\centering
	\includegraphics[width=0.5\textwidth]{../figures/sbs_hw.pdf}
	\caption{System-level hardware architecture with scalable number of heterogeneous PUs: \emph{Spike}, \emph{Conv}, \emph{Pool}, and \emph{FC}}
	\label{fig:hw_sbs}
\end{figure}

\subsection{Conv processing unit}
This hardware module computes the IP dynamics defined by \equ{eq:sbs_update} and offers two modes of operation: \emph{configuration} and \emph{computation}.

\subsubsection{Configuration mode}
In this mode of operation, the PU receives and stores in on-chip memory (BRAM) the \REVIEW{hyperparameters} to compute the IP dynamics: $\epsilon$ as the epsilon, $N$ as the length of $\vec{h}_\mu\in\mathbb{R}^{N}$, $K\in\mathbb{N}$ as the size of the convolution kernel, and $H\in\mathbb{N}$ as the number of IPs to process per transaction. $H$ is the number of IPs forming a layer or a partition.

Additionally, the processing unit also stores in on-chip memory (BRAM) the synaptic weight matrix using a number representation with a reduced memory footprint. Fundamentally, the synaptic weight matrix is defined by $W\in\mathbb{R}^{K\times K\times M\times N}$ with $0\le W(s_t|j)\le1$ and $\sum_{j=0}^{N-1}W(s_t|j)=1$ \cite{rotermund2019Backpropagation}. Hence, $W$ employs only positive normalized real numbers. Therefore, $W$ is deployed using a reduced floating-point or logarithmic representation as follows:

\begin{itemize}
	\item{Custom floating-point representation}.
	In this case, $W$ is deployed with a reduced floating-point representation using the user defined bit-width for the exponent and for the mantissa. For example, 4-bit exponent, 1-bit mantissa; as a result: 5-bit custom floating-point. \REVIEW{The methodology to determine the required bit-width is described in Section~{\ref{sec:dot-product_hardware_module}}.}
	\item{Logarithmic representation}.
	In this case, the synaptic weight matrix is $W\in\mathbb{N}^{K\times K\times M\times N}$ with positive natural numbers. Since $0\le W(s_t|j)\le1$ and $\sum_{j=0}^{N-1}W(s_t|j)=1$, $W$ has only negative values in the logarithmic domain. Hence, the sign bit is omitted, and the values are represented in its positive form. Therefore, $W$ is deployed with a representation using the necessary bit-width for the exponent according to the given application. For example, 4-bit exponent. \REVIEW{The methodology to determine the required bit-width is described in Section~{\ref{sec:dot-product_hardware_module}}.}
\end{itemize}

In order to deploy different SbS network models, the \emph{Conv} processing units can be configured with different synaptic weight matrices and \REVIEW{hyperparameters} as required through the embedded software.

\subsubsection{Computation mode}
In this mode of operation, the PU executes a transaction to process a group of IPs using the previously given \REVIEW{hyperparameters} and synaptic weight matrix. This process operates in six stages as shown in \fig{fig:hw_conv}. In the first two stages, the PU receives $\vec{h}_\mu\in\mathbb{R}^{N}$, then the PU calculates the emitted spike, and stores it in $S^{new}\in\mathbb{N}^{H}$ (output spike vector). From the third to the fifth stage, the PU receives $S_t\in\mathbb{N}^{K\times K}$ (input spike matrix), then it computes the update dynamics, and then it dispatches $\vec{h}_\mu^{new}\in\mathbb{R}^{N}$ (updated IP). This process repeats for $H$ number of loops (for each IP of the layer or partition). Finally, the $S^{new}$ is dispatched.

The computation of the update dynamics (see \fig{fig:hw_conv}(d)) operates in two modular stages: \emph{dot-product} and \emph{neuron update}. First, the \emph{dot-product} module calculates the sum of pairwise products of $\vec{h}_{\mu}$ and $\vec{W}(s_t)$, each pairwise product is stored as intermediate results. Subsequently, the \emph{neuron update} module calculates \equ{eq:sbs_update} reusing previous results and parameters.


The calculation of the dot-product of \equ{eq:sbs_update} represents a considerable computational cost using standard floating-point in non-quantized network models. Fortunately, the pair product of $h_{\mu}(j)$ and $W(s_t|j)$ was defined by us as an approximable factor in the dot-product of \equ{eq:sbs_update}. In the following section, we focus on an optimized dot-product hardware design based on approximate computing.


\begin{figure}[t!]
	\centering
	\includegraphics[width=0.5\textwidth]{../figures/sbs_conv.pdf}
	\caption{The \emph{Conv} processing unit and its six stages: (a) receive IP vector, (b) spike firing, (c) receive spike kernel, (d) update dynamics, (e) dispatch new IP vector, (f) dispatch output spike matrix.}
	\label{fig:hw_conv}
\end{figure}

\subsection{dot-product hardware module}
\label{sec:dot-product_hardware_module}
This dot-product hardware module is part of an application-specific architecture optimized to approximate the dot-product of arbitrary length, see \equ{eq:dot_product}. For quality configurability, we parameterized the mantissa bit-width of $\vec{W}(s_t)$, which provides a tunable trade-off between resource utilization and QoR. Since the lower-order bits have smaller significance than the higher-order bits, removing them may have only a minor impact on QoR. We designate this as hybrid custom floating-point approximation \REVIEW{(see {\fig{fig:product_unit_bitwidth}}(a))}.

\begin{eqnarray} \label{eq:dot_product}
r_{\mu}\left(s_t\right)=\sum_{j=0}^{N-1}h_{\mu}(j)W(s_t|j)
\end{eqnarray}

Further on, we remove the mantissa bits completely in order to use only the exponent of a floating-point representation. Hence, the worst-case quality and yet the most efficient configuration becomes a logarithmic representation. Consequently, this structure leads to advantageous architectural optimizations using only adders and barrel shifters for dot-product approximation in hardware. We designate this as hybrid logarithmic approximation \REVIEW{(see {\fig{fig:product_unit_bitwidth}}(b))}.

\REVIEW{
In order to determine the required bit-width for the number representation, we use {\equ{eq:exp_max}}, {\equ{eq:bits_exp}}, and {\equ{eq:bits_bitwidth}}.}

\begin{eqnarray} \label{eq:exp_max}
E_{\min}=\log _2(\min_{\forall i}(W(i)))
\end{eqnarray}

\begin{eqnarray} \label{eq:bits_exp}
N_E=\lceil\log_2(|E_{\min}|)\rceil
\end{eqnarray}

\begin{eqnarray} \label{eq:bits_bitwidth}
N_W=N_E + N_M
\end{eqnarray}

\REVIEW{
The {\equ{eq:exp_max}} obtains the exponent of the minimum entry value in the synaptic weight matrix. Since $0\le W(s_t|j)\le1$ and $\sum_{j=0}^{N-1}W(s_t|j)=1$, $W$ has only negative values in the logarithmic domain; hence, by searching for the smallest value, we obtain the biggest negative exponent ($E_{\min}$). Then, the {\equ{eq:bits_exp}} obtains the necessary bit-width to represent the exponent ($N_E$). Finally, we obtain the total bit-width by incorporating both exponent and mantissa bit-widths in {\equ{eq:bits_bitwidth}}. $N_M$ denotes the mantissa bit-width, this is a knob parameter that is tuned by the designer to trade-off between resource utilization and QoR. The bit-width concept is illustrated in {\fig{fig:product_unit_bitwidth}}.
}

\begin{figure}
\includegraphics[width=\columnwidth]{../figures/dot-product_unit_bitwidth.pdf}
\caption{Dot-product hardware module with (a) hybrid custom floating-point approximation, and (b) hybrid logarithmic approximation.}
\label{fig:product_unit_bitwidth}
\end{figure}

In this section, we will present three pipelined hardware modules with standard floating-point (IEEE 754) computation, hybrid custom floating-point approximation, and hybrid logarithmic approximation.

\subsubsection{Dot-product with standard floating-point computation}
 The hardware module to calculate the dot-product with standard floating-point computation is shown in \fig{fig:dot_product_float}. This diagram presents the hardware blocks and their clock cycle schedule. This module loads both $h_\mu(j)$ and $W(s|j)$ from BRAM, then the PU executes the pairwise product (\fig{fig:dot_product_float}(c)) and accumulation (\fig{fig:dot_product_float}(d)). The intermediate results of $h_\mu(j) W(s_t|j)$ are stored in BRAM for reuse in the neuron update. The latency in clock cycles of this hardware module is defined by \equ{eq:dot_standard_float_latency}, where $N$ is the dot-product length. This latency equation is obtained from the general pipelined hardware latency formula: $L=\left(N-1\right)II+IL$, where $II$ is the initiation interval (\fig{fig:dot_product_float}(a)), and $IL$ is the iteration latency (\fig{fig:dot_product_float}(b)). Both $II$ and $IL$ are obtained from the high-level synthesis analysis. The equation for the latency with standard 32-bit floating-point is:
 \begin{eqnarray} \label{eq:dot_standard_float_latency}
 L_{f32}=10N+9
 \end{eqnarray}
 
In this design, the high level synthesis tool infers computational blocks with considerable latency cost for standard floating-point. In the case of floating-point multiplication (\fig{fig:dot_product_float}(c)), the synthesis infers a hardware block with a latency cost of 5 clock cycles. Theoretically, this block would handle exponents addition, mantissas multiplication, and mantissa correction if needed. Moreover, in the case of floating-point addition (\fig{fig:dot_product_float}(d)), the synthesis infers a hardware block with a latency cost of 9 clock cycles. Seemingly, this block would handle mantissas alignment, addition, and correction if needed. Therefore, the use of standard floating-point in high-level synthesis results in high computational cost, which represents unnecessary overhead in error-tolerant applications.


\begin{figure}[t!]
	\centering
	\includegraphics[width=0.5\textwidth]{../figures/dot_product_float.pdf}
	\caption{Dot-product hardware module with standard floating-point (IEEE 754) computation, (a) exhibits the initiation interval of 10 clock cycles, (b) presents the iteration latency of 19 clock cycles, (c) shows the pairwise product block in dark-gray, and (d) illustrates the accumulation block in light-gray.}
	\label{fig:dot_product_float}
\end{figure}

\subsubsection{Dot-product with hybrid custom floating-point and logarithmic approximation}
 The hardware module to calculate dot-product with hybrid custom floating-point approximation is shown in \fig{fig:dot_product_custom}. In this design, $h_\mu(j)$ uses standard 32-bit floating-point number representation, and $W(s|j)$ uses a positive reduced custom floating-point number representation, where the mantissa bit width is the quality configurability knob. This parameter is tuned by the designer to trade-off between QoR and resource utilization, thus, energy consumption.
 
 As the most efficient setup and yet the worst-case quality configuration, by completely truncating the mantissa of $W(s|j)$ leads to a slightly different hardware architecture using only adders and shifters, which computes the dot-product with hybrid logarithmic approximation. This is shown in \fig{fig:dot_product_log}.
 
Additionally, the exponent bit-width of $W(s|j)$ is a design parameter for efficient resource utilization and it is defined based on the application or deployment needs.
 
 The hybrid custom floating-point and logarithmic approximation designs work in three phases: \emph{Computation}, \emph{Threshold-test}, and \emph{Result normalization}.
 
 \begin{itemize}
 	\item{Phase I, \emph{Computation}}: 
 	\\This phase approximates the magnitude of the dot-product in a denormalized representation. This is calculated in two iterative steps over each vector element: \emph{pairwise product} and \emph{accumulation}, where \emph{pairwise product} is executed either in hybrid custom floating-point or hybrid logarithmic approximation described below.
 	 \begin{itemize}[label={--}]
 	 	\item{Pairwise product}.
 	 	\begin{itemize} [label={--}]
	 		\item{Hybrid custom floating-point approximation}.
	 	 	As shown in \fig{fig:dot_product_custom}(c) in dark-gray, the pairwise product is approximated by adding exponents and multiplying mantissas of both $W(s|i)$ and $h_\mu(i)$. If the mantissa multiplication results in an overflow, then it is corrected by increasing the  exponent and shifting the resulting mantissa by one position to the right. Then we get $h_\mu(j) W(s_t|j)$ as an intermediate result which is stored for future reuse in the neuron update calculation. In this design the pairwise product has a latency of 5 clock cycles.
	 	 	\item{Hybrid logarithmic approximation}.
	 	 	As shown in \fig{fig:dot_product_log}(c) in dark-gray, the pairwise product is approximated by adding $W(s|i)$ to the exponent of $h_\mu(i)$, since $W(s|j)$ values are represented in the logarithmic domain and $h_\mu(j)$ in standard floating-point. In this design the pairwise product has a latency of one clock cycle.
 	 	\end{itemize}
 		\item{Accumulation}. As shown in both \fig{fig:dot_product_custom}(d) and \fig{fig:dot_product_log}(d) in light-gray, first, it is obtained the denormalized representation of $h_\mu(j) W(s_t|j)$ by shifting its mantissa using its exponent as shifting parameter (barrel shifter). Then, this denormalized representation is accumulated to obtain the approximated magnitude of the dot-product.
 	 \end{itemize}
 	The process of pairwise product and accumulation iterates over each element of the vectors. The computation latency is given by \equ{eq:dot_standard_custom_float_latency} for hybrid custom floating-point, and \equ{eq:dot_log_latency} for hybrid logarithmic, where $N$ is the length of the vectors. Both pipelined hardware modules have the same throughput, since both have two clock cycles as initiation interval. 	
 	\begin{eqnarray} \label{eq:dot_standard_custom_float_latency}
 	L_{custom}=2N+11
 	\end{eqnarray} 	
	\begin{eqnarray} \label{eq:dot_log_latency}
 	L_{log}=2N+7
 	\end{eqnarray}
 	
 	\item{Phase II, \emph{Threshold-test}}: \\
	The accumulated denormalized magnitude is tested to be above of a predefined threshold, it must be above zero, since the dot-product is the denominator in \equ{eq:sbs_update}.
 	If passing the threshold, then the next phase is executed. Otherwise the rest of update dynamics is skipped. The threshold-test takes one clock cycle.
 	\item{Phase III, \emph{Result-normalization}}: \\
 	In this phase, the dot-product is normalized to obtain the exponent and mantissa in order to convert it to standard floating-point for later use in the neuron update. The normalization is obtained by shifting the approximated dot-product magnitude in a loop until it is in the form of a normalized mantissa where the iteration count represents the exponent of the dot-product. Each iteration takes one clock cycle.
 	
 \end{itemize}


The total latency of the hardware module with hybrid custom floating-point and hybrid logarithmic approximation is the accumulated latency of the three phases.

The proposed architectures with approximation approach exceeds the performance of the design with standard floating-point. This performance enhancement is achieved by decomposing the floating-point computation into an advantageous handling of exponent and mantissa using intermediate accumulation in a denormalized representation and only one final normalization.

\begin{figure}[t!]
	\centering
	\includegraphics[width=0.5\textwidth]{../figures/dot_product.pdf}
	\caption{Dot-product hardware module with hybrid custom floating-point approximation, (a) exhibits the initiation interval of 2 clock cycles, (b) presents the iteration latency of 13 clock cycles, (c) shows the pairwise product blocks in dark-gray, and (d) illustrates the accumulation blocks in light-gray.}
	\label{fig:dot_product_custom}
\end{figure}

\begin{figure}[t!]
	\centering
	\includegraphics[width=0.5\textwidth]{../figures/dot_product_log.pdf}
	\caption{Dot-product hardware module with hybrid logarithmic approximation, (a) exhibits the initiation interval of 2 clock cycles, (b) presents the iteration latency of 9 clock cycles, (c) shows the pairwise product block in dark-gray, and (d) illustrates the accumulation blocks in light-gray.}
	\label{fig:dot_product_log}
\end{figure}



\section{Experimental results}
\label{sec:experimental_results}
The proposed hardware/software framework is demonstrated on a Xilinx Zynq-7020 SoC (Zybo-Z7 development board). On the PL, we implement the proposed hardware architecture with a clock frequency at $150 MHz$. On the PS, we execute TF Lite Micro (bare-metal) on the ARM Cortex-A9 at $666MHz$ with NEON floating-point unit (FPU)\cite{xilinx2015zynq}.

To evaluate the performance, we build models $A$ and $B$ in TensorFlow, see \Fig{fig:models}. To evaluate \emph{DConv} tensor operation, model $B$ incorporates depthwise separable convolution operations (a depthwise convolution followed by
a pointwise convolution).

$A$ and $B$ are evaluated with the following hardware implementations:
(1) fixed-point, (2) floating-point LogiCORE, (3) hybrid custom floating-point approximation, and (4) hybrid logarithmic approximation.


\begin{figure}[t!]
	\centering
	\includegraphics[width=0.5\textwidth]{../figures/models.pdf}
	\caption{CNN-based models for case study.}
	\label{fig:models}
\end{figure}

\subsection{Hardware implementations}
\begin{enumerate}
\item{Fixed-point}: To evaluate the compute performance on fixed-point, we convert $A$ and $B$ to TF Lite models with 8-bit fixed-point quantization. The compute performance is presented in \tab{tab:performance_fixed_point}. A runtime execution of $A$ is illustrated in \fig{fig:sched_model_a_fixed}. This implementation achieves a peak acceleration of $45.23\times$ in model $A$ at the tensor operation \emph{(4A) Conv}, see \tab{tab:performance_fixed_point}.

\item{Floating-point LogiCORE}: To evaluate the compute performance on floating-point models, we convert $A$ and $B$ to TF Lite without quantization. The compute performance is presented in \tab{tab:performace_float_logicore}.
This implementation achieves a peak acceleration of $9.77\times$ in model $A$ at the tensor operation \emph{(4A) Conv}.

\item{Hybrid custom floating-point approximation}: This implementation presents a peak acceleration of $44.87\times$ in model $A$ at the tensor operation \emph{(4A) Conv}. See \tab{tab:performace_float_hybrid}. This implementation achieves a $4.59\times$ acceleration over the LogiCORE floating-point implementation. The runtime execution of model $B$ with \emph{DConv} tensor operations is illustrated in \fig{fig:sched_model_b_float}.

\item{Hybrid logarithmic approximation}: This implementation is presented for comparison in \fig{fig:sched_model_a_float}, which shows the runtime executions of model $A$ with the proposed floating-point solutions including hybrid logarithmic approximation.
\end{enumerate}

\begin{table}[!htp]\centering
	\caption{Compute performance with fixed-point on model $A$ and $B$.}\label{tab:performance_fixed_point}
	\scriptsize
\begin{tabular}{lrrrrrrr}\toprule
	\multicolumn{2}{c}{\textbf{Tensor operation}} &\textbf{CPU} &\multicolumn{3}{c}{\textbf{TP (fixed-point)}} &\multirow{2}{*}{\textbf{Accel.}} \\\cmidrule{1-6}
	\textbf{Operation} &\textbf{MOP} &\textbf{t (ms)} &\textbf{t (ms)} &\textbf{MOP/s} &\textbf{GOP/W} & \\\midrule
	\multicolumn{2}{c}{\textbf{Model $A$}} & & & & & \\
	(1A) Conv &1.769 &700.22 &55.19 &32.06 &0.23 &\textbf{12.69} \\
	(2A) Conv &37.748 &12,666.91 &297.08 &127.06 &0.93 &\textbf{42.64} \\
	(3A) Conv &18.874 &6,081.01 &142.99 &131.99 &0.97 &\textbf{42.53} \\
	(4A) Conv &18.874 &5,543.77 &122.58 &153.97 &1.13 &\textbf{45.23} & \\\midrule
	\multicolumn{2}{c}{\textbf{Model $B$}} & & & & & \\
	(1B) DConv &0.027 &13.43 &0.63 &43.74 &0.25 &\textbf{21.25} \\
	(2B) Conv &0.196 &129.95 &11.57 &16.98 &0.12 &\textbf{11.23} \\
	(3B) DConv &0.147 &69.18 &3.33 &44.26 &0.25 &\textbf{20.77} \\
	(4B) Conv &1.048 &378.78 &9.96 &105.25 &0.77 &\textbf{38.02} \\
	(5B) Conv &2.359 &694.60 &16.46 &143.22 &1.05 &\textbf{42.20} \\
	\bottomrule
\end{tabular}
\end{table}

\begin{figure}[t!]
	\centering
	\includegraphics[width=0.5\textwidth]{../figures/sched_A_fixed_point.pdf}
	\caption{Compute performance with fixed-point on model $A$.}
	\label{fig:sched_model_a_fixed}
\end{figure}


\begin{table}[!htp]\centering
	\caption{Compute performance with floating-point LogiCORE on models $A$ and $B$.}\label{tab:performace_float_logicore}
	\scriptsize
	\begin{tabular}{lrrrrrrr}\toprule
		\multicolumn{2}{c}{\textbf{Tensor operation}} &\textbf{CPU} &\multicolumn{3}{c}{\textbf{TP (floating-point LogiCORE)}} &\multirow{2}{*}{\textbf{Accel.}} \\\cmidrule{1-6}
		\textbf{Operation} &\textbf{MOP} &\textbf{t (ms)} &\textbf{t (ms)} &\textbf{MOP/s} &\textbf{GOP/W} & \\\midrule
		\multicolumn{2}{c}{\textbf{Model $A$}} & & & & & \\
		(1A) Conv &1.769 &670.95 &120.07 &14.73 &0.21 &\textbf{5.59} \\
		(2A) Conv &37.748 &12,722.13 &1,328.08 &28.42 &0.40 &\textbf{9.58} \\
		(3A) Conv &18.874 &6,094.85 &636.53 &29.65 &0.42 &\textbf{9.58} \\
		(4A) Conv &18.874 &5,564.79 &569.30 &33.15 &0.47 &\textbf{9.77} & \\\midrule
		\multicolumn{2}{c}{\textbf{Model $B$}} & & & & & \\
		(1B) DConv &0.027 &11.51 &1.557 &17.75 &0.23 &\textbf{7.39} \\
		(2B) Conv &0.196 &94.82 &20.487 &9.59 &0.13 &\textbf{4.62} \\
		(3B) DConv &0.147 &58.84 &8.355 &17.64 &0.23 &\textbf{7.04} \\
		(4B) Conv &1.048 &368.66 &40.271 &26.03 &0.37 &\textbf{9.15} \\
		(5B) Conv &2.359 &697.08 &72.981 &32.32 &0.46 &\textbf{9.55} \\
		\bottomrule
	\end{tabular}
\end{table}

\begin{table}[!htp]\centering
	\caption{Compute performance with hybrid custom floating-point approximation on models $A$ and $B$.}\label{tab:performace_float_hybrid}
	\scriptsize
\begin{tabular}{lrrrrrrr}\toprule
	\multicolumn{2}{c}{\textbf{Tensor operation}} &\textbf{CPU} &\multicolumn{3}{c}{\textbf{TP (H. custom floating-point)}} &\multirow{2}{*}{\textbf{Accel.}} \\\cmidrule{1-6}
	\textbf{Operation} &\textbf{MOP} &\textbf{t (ms)} &\textbf{t (ms)} &\textbf{MOP/s} &\textbf{GOP/W} & \\\midrule
	\multicolumn{2}{c}{\textbf{Model $A$}} & & & & & \\
	(1A) Conv &1.769 &670.95 &68.50 &25.83 &0.39 &\textbf{9.8} \\
	(2A) Conv &37.748 &12,722.13 &307.83 &122.63 &1.85 &\textbf{41.33} \\
	(3A) Conv &18.874 &6,094.85 &147.97 &127.55 &1.93 &\textbf{41.19} \\
	(4A) Conv &18.874 &5,564.79 &124.03 &152.17 &2.30 &\textbf{44.87} & \\\midrule
	\multicolumn{2}{c}{\textbf{Model $B$}} & & & & & \\
	(1B) DConv &0.027 &11.51 &1.41 &19.63 &0.27 &\textbf{8.17} \\
	(2B) Conv &0.196 &94.82 &20.34 &9.43 &0.14 &\textbf{4.66} \\
	(3B) DConv &0.147 &58.84 &6.58 &22.41 &0.31 &\textbf{8.94} \\
	(4B) Conv &1.048 &368.66 &12.75 &82.23 &1.24 &\textbf{28.91} \\
	(5B) Conv &2.359 &697.08 &17.14 &137.68 &2.08 &\textbf{40.68} \\
	\bottomrule
\end{tabular}
\end{table}


\begin{figure*}[t!]
	\centering
	\includegraphics[width=\textwidth]{../figures/sched_A_float_all.pdf}
	\caption{Compute performance with the proposed floating-point solutions on model $A$.}
	\label{fig:sched_model_a_float}
\end{figure*}


\begin{figure}[t!]
	\centering
	\includegraphics[width=0.5\textwidth]{../figures/sched_B_float.pdf}
	\caption{Compute performance on model $B$ (floating-point).}
	\label{fig:sched_model_b_float}
\end{figure}

\subsection{Classification accuracy}

We evaluate the classification accuracy of the CNN models under the effects of custom floating and logarithmic quantization. \tab{tab:formats} presents the list of custom formats proposed for evaluation. In this case, the \emph{filter} and \emph{bias} tensors are quantized from base floating-point representation (IEEE 754) into custom reduced formats with bit-truncation and -rounding methods. For this evaluation, we train $A$ an $B$ for image classification with CIFAR-10 dataset. We deploy the models with a baseline accuracy of 76.6\% for $A$, and 68.8\% for $B$. See \fig{fig:accuracy}.

\begin{table}[!htp]\centering
	\caption{Implemented floating-point formats for accuracy evaluation.}\label{tab:formats}
	\scriptsize
	\begin{tabular}{lrrrrr}\toprule
		\multicolumn{5}{c}{\textbf{Floating-point formats}} \\\cmidrule{1-5}
		\textbf{Name} &\textbf{Size (bits)} &\textbf{Sign} &\textbf{Exponent} &\textbf{Mantissa} \\\midrule
		Logarithmic &6 &1 &5 &0 \\
		S1-E5-M1 &7 &1 &5 &1 \\
		S1-E5-M2 &8 &1 &5 &2 \\
		S1-E5-M3 &9 &1 &5 &3 \\
		S1-E5-M4 &10 &1 &5 &4 \\
		Float16 &16 &1 &5 &10 \\
		BFloat16 &16 &1 &8 &7 \\
		Tensor Float &19 &1 &8 &10 \\
		Float32 &32 &1 &8 &23 \\
		\bottomrule
	\end{tabular}
\end{table}

\begin{figure}[t!]
	\centering
	\includegraphics[width=0.5\textwidth]{../figures/all_models_accuracy.pdf}
	\caption{Accuracy performance using hybrid custom floating-point approximation with various formats. Samples: CIFAR-10 test dataset ($10,000$ images).}
	\label{fig:accuracy}
\end{figure}

\subsection{Resource utilization and power dissipation}
The resource utilization and power dissipation of the TP is listed in \tab{tab:resource}. The power dissipation of the Zynq device is presented in \fig{fig:power}. 

\begin{table}[!htp]\centering
	\caption{Resource utilization and power dissipation of the proposed TP engines.}\label{tab:resource}
	\scriptsize
\begin{tabular}{lrrrrrr}\toprule
	\textbf{} &\multicolumn{4}{c}{\textbf{Post-implementation resource utilization}} &\multirow{2}{*}{\textbf{Power (W)}} \\\cmidrule{2-5}
	\textbf{TP engine} &\textbf{LUT} &\textbf{FF} &\textbf{DSP} &\textbf{BRAM 18K} & \\\midrule
	\multicolumn{6}{l}{\textbf{1) Fixed-point}} \\
	Conv &5,677 &4,238 &78 &70 &0.136 \\
	DConv &7,232 &5,565 &106 &70 &0.171 \\
	Conv + DConv &12,684 &8,015 &160 &70 &0.248 & \\\midrule
	\multicolumn{6}{l}{\textbf{2) Floating-point LogiCore}} \\
	Conv &4,670 &3,909 &59 &266 &0.070 \\
	DConv &6,263 &5,264 &82 &266 &0.075 \\
	Conv + DConv &10,871 &7,726 &123 &266 &0.119 \\\midrule
	\multicolumn{6}{l}{\textbf{3 ) Hybrid custom floating-point approximation}} \\
	Conv &6,787 &4,349 &56 &74 &0.066 \\
	DConv &8,209 &5,592 &79 &74 &0.072 \\
	Conv + DConv &14,590 &8,494 &117 &74 &0.108 & \\\midrule
	\multicolumn{6}{l}{\textbf{4) Hybrid logarithmic approximation}} \\
	Conv &6,662 &4,242 &54 &58 &0.060 \\
	DConv &8,110 &5,380 &77 &58 &0.066 \\
	Conv + DConv &14,370 &8,175 &113 &58 &0.105 \\
	\bottomrule
\end{tabular}
\end{table}

\begin{figure}[t!]
	\centering
	\includegraphics[width=0.5\textwidth]{../figures/power_breackdown.pdf}
	\caption{Estimated power dissipation of the Zynq-7020 SoC with different TP engines.}
	\label{fig:power}
\end{figure}


\subsection{Discussion}
\begin{enumerate}
	\item{Acceleration}: The dot-product with hybrid custom floating-point approximation achieves better performance with larger vectors. Since \emph{DConv} performs a channel-wise spatial dot-product, this implementation presents very limited acceleration compared with the LogiCORE-based implementation.
	
	\item{Energy}: The implementations with hybrid custom floating-point and logarithmic approximation are the most efficient with energy reduction of $954\times$ and $1,055\times$, respectively. \tab{tab:edp} presents the energy-delay product (EDP) and energy reduction in \emph{(4A) Conv} operator.
	
\begin{table}[!htp]\centering
	\caption{Energy consumption in tensor operation \emph{(4A) Conv}.}\label{tab:edp}
	\scriptsize
	\begin{tabular}{lrrrrr}\toprule
		\textbf{Engine} & \textbf{t (ms)} &\textbf{Power (W)} &\textbf{EDP (J)} &\textbf{Reduction} \\\midrule
		CPU &5,564.79 &1.404 &7,812.97 &1.00 \\
		Fixed-point &122.58 &0.136 &16.67 &468.66 \\
		Floating-point LogiCORE &569.30 &0.070 &39.85 &196.05 \\
		Hybrid custom floating-point &124.03 &0.066 &8.19 &\textbf{954.43} \\
		Hybrid logarithmic &123.32 &0.060 &7.40 &\textbf{1,055.92 }\\
		\bottomrule
	\end{tabular}
\end{table}
	
	\item{Resource utilization}: In the case of fixed-point implementation, the TP includes additional logic to support the quantized multiplier and shifter channel-wise corrections implemented for TF Lite 8-bit quantization. This TP presents the highest power dissipation.
	
	\item{Accuracy}: Based on the presented classification accuracy, the hybrid custom floating-point approximation presents the best trade off between QoR and energy-efficiency.
	
	Theres is no accuracy degradation when using fixed-point and floating-point LogiCORE IP.
\end{enumerate}
\section{Conclusions}
\label{sec:conclusions}

In this paper, we present a tensor processor as a dedicated hardware accelerator for TensorFlow Lite on embedded FPGA. We accelerate Conv2D and DepthwiseConv2D tensor operations with fixed-point and floating-point. The proposed compute optimization performs vector dot-product with hybrid custom floating-point and logarithmic approximation. This approach accelerates computation, reduces energy consumption and resource utilization. To demonstrate the potential of the proposed architecture, we presented a design exploration with four compute engines: (1) fixed-point, (2) Xilinx floating-point LogiCORE IP, (3) hybrid custom floating-point approximation, and (4) hybrid logarithmic approximation.

A single tensor processor running at 150 MHz on a Xilinx Zynq-7020 achieves 45X runtime acceleration and 951X power reduction on Conv2D tensor operation compared with ARM Cortex-A9 at 666MHz, and 4.5X compared with the equivalent implementation with floating-point LogiCORE IP.

\section * {Acknowledgments}\label{sec:Ack}
This work is funded by the \textit{Consejo Nacional de Ciencia y Tecnologia -- CONACYT} (the Mexican National Council for Science and Technology).


\bibliographystyle{IEEEtran}
\bibliography{../content/bibliography.bib}

\end{document}
