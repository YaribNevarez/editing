\section{Related work}
\label{sec:related_work}
%%%%%%%%%%%%%%%%%%%%%%%%%%%%%%%%%%%%%%%%%%%%%%%%%%%%%%
Recently, some state of the art survey on hardware architectures for SNN have been reported \cite{Design_Exploration_SbS_Trans20, SNN_Survey_Trans19}. In particular, Nassim Abderrahmane et al. briefly describe and compare some recent implementations of ASIC and FPGA where only two are suitable for embedded systems. As a typical example of the current state of the art, Furber et al., presents SpiNNaker \cite{Spinnaker_Trans13}, aiming to simulate very large SNNs in real-time. It is composed of 48 chips containing a shared memory and 18 ARM cores with small local memory each processor. The main feature of SpiNNaker are the support for several neuron models, synaptic plasticity rules, incremental learning capabilities and efficient communication system. This architecture is suitable for neuroscience research but not for embedded applications. Further on, in a previous research Rotermund et al., demonstrated the feasibility of a neuromorphic SbS IP in a Xilinx Virtex 6 FPGA \cite{rotermund2018massively}. It provides a massively parallel architecture, optimized for memory access and suitable for ASIC implementations. However, this design is considerably resource-demanding to be deployed as a full and functional SbS network in the current embedded technology.

Beside the actual architectures, researches have also identified design methodologies as a critical problem for the efficient development of SNN \cite{Design_Exploration_SbS_Trans20}. For example, Nassim Abderrahmane et al., develop a behavioral level simulator for neuromorphic hardware architectural exploration named NAXT, capable to reduce the number of spikes while keeping the neuron's model resulting in lower power consumption. This work provides a great exploration of SNN for different network topologies and computation approaches on NAXT. However, this simulator presents 62\% of accuracy on MNIST classification task.
%%%%%%%%%%%%%%%%%%%%%%%%%%%%%%%%%%%%%%%%%%%%%%%%%%%%%%