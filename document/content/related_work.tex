\section{Related work}
\label{sec:related_work}
%%%%%%%%%%%%%%%%%%%%%%%%%%%%%%%%%%%%%%%%%%%%%%%%%%%%%%
Recently, some state of the art hardware architecture surveys have been reported for SNN \cite{Design_Exploration_SbS_Trans20, SNN_Survey_Trans19, zhang2018survey}. Nassim Abderrahmane et al. briefly describe and compare some recent ASIC and FPGA implementations where only two are suitable for embedded systems. In \cite{Spinnaker_Trans13}, Furber et al. present SpiNNaker, which is suitable for neuroscience research but not for embedded applications.


Further on, in earlier research Rotermund et al. demonstrated the feasibility of a neuromorphic SbS IP on a Xilinx Virtex 6 FPGA \cite{rotermund2018massively}. It provides a massively parallel architecture, optimized for memory access and suitable for ASIC implementations. However, this design is considerably resource-demanding to be implemented as a complete and functional SbS network in today's embedded technology.

In \cite{nevarez2020accelerator}, we presented a cross-platform accelerator framework for design exploration and testing of fully functional SbS network models in embedded systems. As a software-hardware solution, this framework offers a comprehensive high level software API that allows the construction of scalable sequential models with configurable hardware acceleration.

%%%%%%%%%%%%%%%%%%%%%%%%%%%%%%%%%%%%%%%%%%%%%%%%%%%%%%