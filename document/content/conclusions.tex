\section{Conclusions}
\label{sec:conclusions}
In this publication, we accelerate SbS neural networks with a dot-product functional unit based on approximate computing. This approach reduces computational latency, memory footprint, and power dissipation while preserving accuracy and noise robustness. Additionally, the proposed design is compatible with standard floating-point and adaptable as a building block for other error-resilient applications.

We demonstrate our approach addressing a design exploration flow on a Xilinx Zynq-7020 with a deployment of NMIST classification task, this implementation achieves up to $20.49\times$ latency enhancement, $8\times$ synaptic memory footprint reduction, less than $0.5\%$ of accuracy degradation, with a $12.35\%$ of energy efficiency improvement over the standard floating-point hardware implementation. Furthermore, with positive additive uniformly distributed noise at $50\%$ of amplitude on the input image, the SbS network simulation presents an accuracy degradation of less than $5\%$. As application-specific quality metric, the resulting noise robustness demonstrates a sufficient QoR for minimal impact on the overall accuracy of the neural network using the proposed approach. These results suggest available room for further and more aggressive approximation techniques.

In conclusion, based on the relaxed need for fully accurate or deterministic computation of SbS neural networks, approximate computing techniques allow substantial enhancement in processing efficiency with moderated accuracy degradation.

\section * {Acknowledgments}\label{sec:Ack}
This work is funded by the \textit{Consejo Nacional de Ciencia y Tecnologia -- CONACYT} (the Mexican National Council for Science and Technology).
