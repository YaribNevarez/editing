\section{Conclusions}
\label{sec:conclusions}
In this publication, we accelerate SbS neural networks with a dot-product functional unit based on approximate computing, this approach reduces computational latency, memory footprint, and power dissipation while preserving inference accuracy. For output quality monitoring, we propose the noise tolerance plot as an intuitive visual model to provide insights into the accuracy degradation of SbS networks under approximate processing effects, this plot revels inherent error resilience, hence, potential for approximation allowance.

We demonstrate our approach addressing a design exploration flow on a Xilinx Zynq-7020 with a deployment of NMIST classification task, this implementation achieves up to $20.49\times$ latency enhancement, $8\times$ synaptic memory footprint reduction, less than $0.5\%$ of accuracy degradation, with a $12.35\%$ of energy efficiency improvement over the standard floating-point hardware implementation. Furthermore, with positive additive uniformly distributed noise at $50\%$ of amplitude on the input image, the SbS network simulation presents an accuracy degradation of less than $5\%$. As output quality monitor, the resulting noise tolerance plots demonstrate a sufficient QoR for minimal impact on the overall accuracy of the neural network under the effects of the proposed approximation technique. These results suggest available room for further and more aggressive approximate processing approaches.

In conclusion, based on the relaxed need for fully accurate or deterministic computation of SbS neural networks, approximate computing techniques allow substantial enhancement in processing efficiency with moderated accuracy degradation.

\section * {Acknowledgments}\label{sec:Ack}
This work is funded by the \textit{Consejo Nacional de Ciencia y Tecnologia -- CONACYT} (the Mexican National Council for Science and Technology).
