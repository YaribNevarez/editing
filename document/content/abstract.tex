\title {Accelerating Spike-by-Spike Neural Networks with Approximate Dot-Product on FPGA}

\author{
	\uppercase{Yarib Nevarez}\authorrefmark{1},	
	\uppercase{David Rotermund}\authorrefmark{2},
	\uppercase{Klaus R. Pawelzik}\authorrefmark{3},
	\uppercase{Alberto Garcia-Ortiz}\authorrefmark{4} \IEEEmembership{Member, IEEE},
}

\address[1]{Institute of Electrodynamics and Microelectronics, University of Bremen, Bremen 28359, Germany (e-mail: nevarez@item.uni-bremen.de)}

\address[2]{Institute for Theoretical Physics, University of Bremen, Bremen 28359, Germany (e-mail: davrot@@neuro.uni-bremen.de)}

\address[3]{Institute for Theoretical Physics, University of Bremen, Bremen 28359, Germany (e-mail: pawelzik@@neuro.uni-bremen.de)}

\address[4]{Institute of Electrodynamics and Microelectronics, University of Bremen, Bremen 28359, Germany (e-mail: agaracia@item.uni-bremen.de)}

\tfootnote{This work is funded by the Consejo Nacional de Ciencia
	y Tecnologia - CONACYT (the Mexican National Council for
	Science and Technology)}

\markboth
{Author \headeretal: Preparation of Papers for IEEE TRANSACTIONS and JOURNALS}
{Author \headeretal: Preparation of Papers for IEEE TRANSACTIONS and JOURNALS}

\corresp{Corresponding author: Yarib Nevarez (e-mail: nevarez@item.uni-bremen.de).}
.

\begin{abstract}
The Spike-by-Spike (SbS) neural network algorithm is a powerful machine learning (ML) technique for image classification with an extraordinary noise robustness. However, deep SbS networks are highly compute and data intensive, requiring new approaches to improve deployment efficiency in resource limited-devices. In this paper, we accelerate SbS neural networks with a dot-product hardware design based on approximate computing, which leverages the intrinsic error-resilience of neural networks. This approach is compatible with standard floating-point, and it does not require training algorithm adjustments, representing a convenient solution for efficient deployment of emerging ML algorithms on embedded applications. To demonstrate our approach, we address a design exploration flow using high-level synthesis and a Xilinx FPGA. As a result, the proposed design achieves up to $20.49\times$ latency enhancement, $8\times$ synaptic memory footprint reduction, and less than $0.5\%$ of accuracy degradation on handwritten digit recognition task.
	
\end{abstract}

\begin{keywords}
Artificial intelligence, spiking neural networks, approximate computing, logarithmic, parameterisable floating-point, optimization, hardware accelerator, embedded systems, FPGA
\end{keywords}

\titlepgskip=-15pt

\maketitle
