\title {Accelerating Spike-by-Spike Neural Networks \added{on FPGA} with Hybrid Custom Floating-Point and Logarithmic Dot-Product Approximation \deleted{on FPGA}}

\author{
	\uppercase{Yarib Nevarez}\authorrefmark{1},	
	\uppercase{David Rotermund}\authorrefmark{2},
	\uppercase{Klaus R. Pawelzik}\authorrefmark{3},
	\uppercase{Alberto Garcia-Ortiz}\authorrefmark{4} \IEEEmembership{Member, IEEE},
}

\address[1]{Institute of Electrodynamics and Microelectronics, University of Bremen, Bremen 28359, Germany (e-mail: nevarez@item.uni-bremen.de)}

\address[2]{Institute for Theoretical Physics, University of Bremen, Bremen 28359, Germany (e-mail: davrot@@neuro.uni-bremen.de)}

\address[3]{Institute for Theoretical Physics, University of Bremen, Bremen 28359, Germany (e-mail: pawelzik@@neuro.uni-bremen.de)}

\address[4]{Institute of Electrodynamics and Microelectronics, University of Bremen, Bremen 28359, Germany (e-mail: agaracia@item.uni-bremen.de)}

\tfootnote{This work is funded by the Consejo Nacional de Ciencia
	y Tecnologia - CONACYT (the Mexican National Council for
	Science and Technology)}

\markboth
{Author \headeretal: Preparation of Papers for IEEE TRANSACTIONS and JOURNALS}
{Author \headeretal: Preparation of Papers for IEEE TRANSACTIONS and JOURNALS}

\corresp{Corresponding author: Yarib Nevarez (e-mail: nevarez@item.uni-bremen.de).}

\begin{abstract}
  Spiking neural networks (SNNs) represent a promising alternative to
  conventional neural networks. In particular, the so called
  Spike-by-Spike (SbS) neural networks provide an exceptional noise
  robustness and a reduced complexity. However, deep SbS networks
  require a memory footprint and a computational cost unsuitable for
  embedded applications. To address this problem, this work exploits
  the intrinsic error-resilience of neural networks to improve
  performance and to reduce the hardware complexity. More precisely,
  we design a dot-product hardware unit based on approximate computing
  with \replaced{configurable quality}{quality configurable design}
  using hybrid custom floating-point and logarithmic number
  representation. This approach reduces computational latency, memory
  footprint, and power dissipation while preserving inference
  accuracy. To demonstrate our approach, we address a design
  exploration flow using high-level synthesis and a Xilinx SoC-FPGA.
  \replaced{The}{As a result, the} proposed design \replaced{reduces
    $20.5\times$ computational latency and $8\times$ weight memory
    footprint, with}{ achieves $20.5\times$ latency enhancement,
    $8\times$ weight memory footprint reduction, and} less than
  $0.5\%$ of accuracy degradation on \added{a} handwritten digit
  recognition task.
	
\end{abstract}

\begin{keywords}
Artificial intelligence, spiking neural networks, approximate computing, logarithmic, parameterisable floating-point, optimization, hardware accelerator, embedded systems, FPGA
\end{keywords}

\titlepgskip=-15pt

\maketitle
